\documentclass[french]{supaero-ria}

\usepackage[utf8]{inputenc}
\usepackage{booktabs}
\usepackage{bibtopic}
\usepackage{url}
\usepackage[pdftex]{hyperref}

\begin{document}

\author{Christophe Garion}
\dep{DMIA}
\ur{MARS}
\formation{SUPAERO}
\uf{IN}

\date{2009}

\newens{Rien de bien nouveau sous le soleil.}

\modules{

  \subsubsection{Responsabilité de modules et enseignement}
  \label{sec:resp-de-modul}

  \hspace{1cm}\begin{tabular}[h]{l l l l }
    \toprule[1.5pt]
    Code & Intitulé & Volume & Commentaires\\
    \midrule
    IN600 & \LaTeX avancé & 40h & en collaboration avec EZ\\
    IN001 & Comment allumer un ordinateur? & 40h & \\
    \bottomrule[1.5pt]
  \end{tabular}

  \subsubsection{Participation à modules}
  \label{sec:part-modul}

  \hspace{1cm}\begin{tabular}[h]{l l l l }
    \toprule[1.5pt]
    Code & Intitulé & Volume & Commentaires\\
    \midrule
    IN732 & Super compliqué & 40h & je n'ai rien fait pour le
    module, mais j'ai participé à une réunion\\
    \bottomrule[1.5pt]
  \end{tabular}
}

\encadrement{J'ai effectivement fait travailler les élèves comme des
  esclaves pour mon intérêt personnel (lavage de voiture, courses etc.)}

\ingenierie{J'ai beaucoup ingénieré cette année.}

\wishens{J'aimerais bien ne plus avoir de copies à corriger dorénavant.}

\actrech{J'ai beaucoup cherché, peu trouvé.}

\animation{Participation importante aux pots de thèses, d'HDR et de
  départs à la retraite.}

\pub{

\nocite{garion09a,garion09b,garion09c}

\begin{btSect}{review}
  \subsection*{Publications avec comité de lecture familial}
  \btPrintCited
\end{btSect}

\begin{btSect}{woreview}
  \subsection*{Publications sans comité de lecture}
  \btPrintCited
\end{btSect}
}

\whatelse{J'aime bien jouer au football.}

\misc{ Si on ne remplit pas un champ, comme par exemple
  \texttt{newens} si on n'a pas de nouveautés en ce qui concerne
  l'enseignement, celui-ci n'apparaîtra pas dans le RIA.

  Comme pour les autres classes (CR de réunion, notes, lettre), on
  peut utiliser un fichier de configuration avec l'extension
  \texttt{.sup} et passer son nom en option de \texttt{documentclass}.

  Pour faire une bibliographie par thème, j'utilise le paquetage
  \texttt{bibtopic}. Il ne faut pas alors oublié d'appeler
  \texttt{bibtex} sur les fichiers \texttt{.aux} supplémentaires
  produits. On peut également utiliser l'utilitaire \texttt{bib2bib}
  pour produire différents fichiers BibTeX par types de publication,
  cf. \url{http://www.lri.fr/~filliatr/bibtex2html/}.  
}

\body

\end{document}
