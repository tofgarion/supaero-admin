\documentclass[french]{supaero-orga}

\usepackage[utf8]{inputenc}
\usepackage{url}
\usepackage[T1]{fontenc}
\usepackage{ae,aecompl,aeguill}
\usepackage{booktabs}
\usepackage[pdfversion=1.7, pdftex, colorlinks, urlcolor=blue, pdfstartview=FitH, pdfhighlight={/N}]{hyperref}
\usepackage{eurosym}
\def\EUR{{\euro}}
\usepackage{verbatim}

\begin{document}

\author{C. Garion, p/o A. Gamisans}

\title[Accueil de PDSAG 2012]{Demande d'accueil pour
  l'organisation de la conférence PDSAG 2011}
\date{\textbf{25 décembre 2011}}
\lieu{ISAE campus Rangueil, Toulouse, France}
\nbpart{1}
\pbscientifique{
  La conférence aborde deux thèmes:
  
  \begin{itemize}
  \item une augmentation substantielle des émoluments de AG    
  \item la mise à disposition des moyens de l'institut pour la
    réalisation d'une résidence principale dans le nord de Toulouse
  \end{itemize}
}
\moyens{
  \begin{itemize}
  \item augmentation de 15\EUR par mois
  \item mise à disposition de IL pour les travaux de construction de
    la maison de AG
  \end{itemize}
}
\maketitle

\section{Présentation de la conférence}
\label{sec:presentation}

La conférence PDSAG (Plus De Sous pour Ausias Gamisans) est une
conférence annuelle qui aura lieu cette année le \textbf{25 décembre
  2011}. L'objectif de la conférence est d'assurer une année de plus
un niveau de vie nababesque à notre collègue Ausias Gamisans: révision
de sa Twingo\footnote{Vendue à Ausias Gamisans par Emmanuel Zenou, de
  nombreuses anomalies ont été décelées sur le véhicule depuis la
  vente: compteur falsifié, numéro d'identification du moteur limé
  etc.}, crêche pour sa fille etc. Pour l'édition 2012, comme les
éditions précédentes, PDSAG sera localisée à l'ISAE campus
Rangueil. La conférence se veut avant tout un lieu d'échange (de
monnaie sonnante et trébuchante).

La soumission s'effectuera sous forme de chèques (sur présentation
d'une pièce d'identité) ou d'espèces. Aucun retour ne sera effectué.

\subsection{Comités de programme et d'organisation de PDSAG}
\label{sec:comit}

Les présidents des comités sont marqués en gras.

\subsubsection{Comité de programme PDSAG}
\label{sec:comite-scient-pdsag}

\begin{center}
  \begin{tabular}{l l}
    \toprule[1.5pt]
    \textbf{Nom} & \textbf{Affiliation}\\
    \midrule
    \textbf{Ausias Gamisans} & \textbf{ISAE, Toulouse} \\ 
    \bottomrule[1.5pt]
  \end{tabular}
\end{center}

\subsection{Comité d'organisation}
\label{sec:comite-dorganisation}

\begin{center}
  \begin{tabular}{l l}
    \toprule[1.5pt]
    \textbf{Nom} & \textbf{Affiliation}\\
    \midrule
    \textbf{Ausias Gamisans} & \textbf{ISAE, Toulouse}\\
    \bottomrule[1.5pt]
  \end{tabular}
\end{center}

\subsection{Intérêt pour l'ISAE}
\label{sec:interet-pour-lisae}

Aucun.

\section{Demandes spécifiques faites à l'ISAE}
\label{sec:demand-spec-lisae}

Outre l'accueil de la conférence le 25 décembre 2011 sur le site de
Rangueil, nous souhaiterions bénéficier d'un certain nombre de
services offerts par l'ISAE. Les demandes sont listées sur le
tableau~\ref{tab:demandes}, ainsi que le ou les services associés.

\begin{table}[h]
  \centering
  \begin{tabular}[h]{p{0.33\textwidth} l p{0.33\textwidth}}
    \toprule[1.5pt]
    \textbf{Demande} & \textbf{Service(s) concerné(s)} & \textbf{Remarques}\\
    \midrule
    mise à disposition de 15000\EUR & AC & sans remboursement
    possible. En coupures usagées de 10\EUR\\

    mise à disposition des véhicules de l'ISAE & IL/LOG & un véhicule
    de type Trafic pour la durée des travaux + un véhicule de type
    Clio pour usage personnel \\
    
    mise à disposition des personnels électriciens &
    IL/MAIN & pour la mise aux normes de la résidence principale de AG\\

    réalisation d'un visuel vantant les mérites de la formation en LV
    & COM 
    & 3m par 2m en quadrichromie avec affichage dans tout l'institut\\
    \bottomrule[1.5pt]
  \end{tabular}
  \caption{Demandes spécifiques faites à l'ISAE pour l'organisation
    de PDSAG 2011}
  \label{tab:demandes}
\end{table}

Nous sollicitons l'ISAE pour prendre en charge gracieusement tous les
frais inhérents à la conférence. Le budget prévisionnel de la
conférence est donné en section~\ref{sec:budget-previsionnel}.

\section{Affichage et visibilité de l'ISAE}
\label{sec:affich-et-visib}

En contrepartie, AG remerciera personnellement le Directeur Général de
l'ISAE.

\section{Budget prévisionnel}
\label{sec:budget-previsionnel}

Le budget prévisionnel de la conférence est donné sur le
tableau~\ref{tab:budget}.

\begin{table}[h]
  \centering
  \begin{tabular}[h]{l r r}
    \toprule[1.5pt]
    \textbf{nature} & \textbf{dépenses} & \textbf{recettes}\\
    \midrule
    augmentation salariale AG & 15000 & \\
    construction garage habitation principale AG & 10000\\
    électricité habitation principale AG & 5000\\
    fournitures et services divers & 5000\\
    \midrule
    subvention ISAE en liquide & & 15000\\
    mise à disposition IL & & 20000\\
    \midrule
    \textbf{total} & \textbf{35000} & \textbf{35000}\\
    \bottomrule[1.5pt]
  \end{tabular}
  \caption{Budget prévisionnel de la conférence PDSAG 2011}
  \label{tab:budget}
\end{table}

\section{Utilisation de la classe}
\label{sec:utilisation-de-la}

\subsection{Chargement de la classe et dimensions}
\label{sec:chargement-classe}

La classe s'utilise très simplement. Elle est dérivée de la classe
\texttt{article}, donc la plupart des commandes, longueurs, éléments
de mise en page usuels s'y trouvent. Il suffit dans un premier temps
d'utiliser \texttt{supaero-orga} comme classe~:

\begin{verbatim}
  \documentclass{supaero-orga}
\end{verbatim}

Les dimensions de la page ont été définies grâce au paquetage
\texttt{geometry}. On peut donc se servir de ce dernier pour les
changer. Par exemple, pour changer la proportion horizontale de texte
sur la page\footnote{la valeur est 0.85 par défaut.}, on peut faire:

\begin{verbatim}
  \geometry{hscale=0.7}
\end{verbatim}

\subsection{Entêtes et pieds de page}
\label{sec:entetes-et-pieds}

Les entêtes et pieds de page sont définis grâce au paquetage
\texttt{fancyhdr}. On dispose donc de plusieurs commandes pour
modifier directement les éléments~:

\begin{verbatim}
  \lhead{contenu de la partie gauche de l'entête}
  \chead{contenu de la partie centrale de l'entête}
  \rhead{contenu de la partie droite de l'entête}
  \lfoot{contenu de la partie gauche du pied de page}
  \cfoot{contenu de la partie centrale du pied de page}
  \rfoot{contenu de la partie droite du pied de page}
\end{verbatim}

\subsection{Page de titre}
\label{sec:page-de-titre}

La page de titre se personnalise de la façon suivante:

\begin{itemize}
\item la commande \verb!\title{}! permet de donner le titre. On peut
  lui donner en option un titre court qui servira sur les entêtes des
  pages suivantes (par exemple
  \verb!\title[titrecourt]{titresuperlong}!)
\item la commande \verb!\author{}! permet de donner le nom du
  rédacteur du document
\item la commande \verb!\date{}! permet de donner la date de
  l'événement
\item la commande \verb!\lieu{}! permet de donner le lieu de
l'événement
\item la commande \verb!\nbpart{}! permet de donner le nombre attendu
  de participants
\item la commande \verb!\{pbscientifique}! permet de donner la
problématique scientifique de l'événement
\item la commande \verb!\{moyens}! permet de donner les moyens
demandés à l'ISAE
\end{itemize}

\subsection{Fichier de déclarations}
\label{sec:fich-de-decl}

Il se peut que l'on utilise souvent les mêmes éléments. Comme pour les
autres classes, on peut avec la classe \texttt{supaero-orga} utiliser
un fichier \texttt{mon-fichier.sup} qui contient des redéfinitions
(par exemple \verb!\author{}! etc.). Pour cela, il suffit de passer
\verb!mon-fichier! en option de \verb!\documentclass!:

\begin{verbatim}
  \documentclass[mon-fichier]{supaero-orga}
\end{verbatim}

\section{Code source du présent exemple}

\verbatiminput{exempleOrganisationCongres.tex}

\section{Code source de la classe \texttt{supaero-orga}}
\label{sec:code-source}

\verbatiminput{supaero-orga.cls}

\end{document}
