\documentclass[french]{supaero-note-cf}

\usepackage[utf8]{inputenc}
\usepackage{url}
\usepackage[pdftex]{hyperref}
\usepackage{verbatim}

\begin{document}

\title{Suppression des enseignements de Langues Vivantes}
\author{C. Garion}
\date{31 brumaire 2023}
\function{RP1}

\section{Objectifs}
\label{sec:objectifs}

L'objectif de cette réforme est de supprimer complétement les
enseignements de langues vivantes à l'ISAE.

\section{Justifications}
\label{sec:justifications}

Abreuvés de séries américaines débiles en VO dès le plus jeune âge,
les élèves sont bilingues anglais dès l'âge de 5 ans. Quant à
l'apprentissage éventuel d'une autre langue, cela ne sert strictement
à rien. L'aspect «~culturel~» avancé par le Professeur Honoraire
Gamisans est ridicule: nous formons des ingénieurs, on leur fait faire
de la science et pis c'est tout.

\section{Mise en place}
\label{sec:mise-en-place}

La réforme sera mise en place en trois étapes:

\begin{itemize}
\item plasticage des labos langues et des bureaux des professeurs de
  langues (si possible quand ils sont là)
\item licenciement des professeurs de langues (j'ai quelques vidéos du
  Professeur Honoraire Gamisans qui permettent de justifier le
  licenciement pour faute grave)
\item récupération des heures pour faire de la science, de la vraie
\end{itemize}

\section{Utilisation de la classe}
\label{sec:utilisation-de-la}

Rien de bien particulier, il y a quelques commandes utiles pour la
classe:

\begin{verbatim}
  \title[titre court pour les entêtes]{titre de la contribution}
  \date{date du CP}
  \author{auteur de la contribution}
  \function{fonction de l'auteur}
\end{verbatim}

Comme pour les autres classes (CR de réunion, notes, lettre), on
peut utiliser un fichier de configuration avec l'extension
\texttt{.sup} et passer son nom en option de \texttt{documentclass}.

\section{Code source de la présente note}
\label{sec:code-source}

Le code source de la présente note est présenté sur le listing
suivant.

\verbatiminput{exempleNoteCF.tex}

\section{Code source de la classe}
\label{sec:code-source-supaero-note(cf)}

\verbatiminput{supaero-note-cf.cls}

\end{document}