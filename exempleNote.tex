\documentclass[fr,biblatex]{supaero-note}

\usepackage[utf8]{inputenc}
\usepackage{url}
\usepackage{verbatim}
\addbibresource{supaero}

\begin{document}

\textrhead{
  \begin{tabular}[b]{l}
    DFS: C. Bérard\\
    RP1: C. Garion
  \end{tabular}
  }

\numero{2009BLABLA}

\objet{Établissement d'un manuel d'utilisation d'une classe \LaTeX
  pour les notes DFS}

\destinataires{
  \begin{itemize}
  \item professeurs DFS
  \end{itemize}
}

\motscles{\LaTeX, yenamarredeword}

\revisions{
   & 1/10/2009 & & Création du document\\
   \hline
}

\author{C. Garion}
\function{E/C DMIA}

\title{Note portant sur l'utilisation d'un style \LaTeX pour les notes DFS}
\maketitle

\section{Objectifs}
\label{sec:objectifs}

L'objectif de ce «~travail~» était de produire une classe \LaTeX
\texttt{supaero-note} permettant de créer des notes de travail pour la
DFS (Direction de Formation SUPAERO). La classe ainsi obtenue devait
être facile d'utilisation et personnalisable.

La classe a été produite en suivant les conseils
de~\cite{flynn07:_rollin_docum_class} et
\cite{hefferson05:_linut}. Deux paquetages principaux ont été
utilisés: \texttt{fancyhdr} et \textsf{geometry} qui sont disponibles
dans toutes les distributions \LaTeX. Le lecteur souhaitant étudier de
plus près ces paquetages pourra consulter les archives
CTAN~\cite{ctan} ou \cite{mittelbach04:_latex_companion}.

\section{Utilisation de la classe}
\label{sec:utilisation-classe}

\subsection{Chargement de la classe et dimensions}
\label{sec:chargement-classe}

La classe s'utilise très simplement. Elle est dérivée de la classe
\texttt{article}, donc la plupart des commandes, longueurs, éléments
de mise en page usuels s'y trouvent. Il suffit dans un premier temps
d'utiliser \texttt{supaero-note} comme classe:

\begin{verbatim}
  \documentclass{supaero-note}
\end{verbatim}

Les dimensions de la page ont été définies grâce au paquetage
\texttt{geometry}. On peut donc se servir de ce dernier pour les
changer. Par exemple, pour changer la proportion horizontale de texte
sur la page\footnote{la valeur est 0.85 par défaut.}, on peut faire:

\begin{verbatim}
  \geometry{hscale=0.7}
\end{verbatim}

\subsection{Entêtes et pieds de page}
\label{sec:entetes-et-pieds}

Les entêtes et pieds de page sont définis grâce au paquetage
\texttt{fancyhdr}. On dispose donc de plusieurs commandes pour
modifier directement les éléments:

\begin{verbatim}
  \lhead{contenu de la partie gauche de l'entête}
  \chead{contenu de la partie centrale de l'entête}
  \rhead{contenu de la partie droite de l'entête}
  \lfoot{contenu de la partie gauche du pied de page}
  \cfoot{contenu de la partie centrale du pied de page}
  \rfoot{contenu de la partie droite du pied de page}
\end{verbatim}

On dispose également d'une commande spéciale pour redéfinir la partie
droite de l'entête de la page de titre dont la valeur par défaut est
«~DFS: C. Bérard~»:

\begin{verbatim}
  \textrhead{nouvelle partie droite de l'entête}
\end{verbatim}

De la même façon, on dispose des commandes \verb!\textlhead! et
\verb!\textchead!.

\subsection{Page de titre}
\label{sec:page-de-titre}

La page de titre se personnalise de la façon suivante:

\begin{itemize}
\item la commande \verb!\title{}! permet de donner le titre. On peut
  lui donner en option un titre court qui servira sur les entêtes des
  pages suivantes (par exemple
  \verb!\title[titrecourt]{titresuperlong}!)
\item la commande \verb!\author{}! permet de donner le nom du
  rédacteur du document
\item la commande \verb!\function{}! permet de donner la fonction du
  rédacteur du document
\item la commande \verb!\numero{}! permet de donner le numéro de la
  note
\item la commande \verb!\objet{}! permet de donner l'objet de la note
\item la commande \verb!\destinataires! permet de donner la liste des
  destinataires de la note
\item la commande \verb!\motscles{}! permet de donner la liste des
  mots-clés
\item la commande \verb!\revisions{}! permet de donner la liste des
  révisions. Chaque révision est en fait une ligne d'un tableau, donc
  il faut utiliser \verb!&! pour séparer les colonnes du tableau,
  \verb!\\! pour passer à une autre ligne et \verb!\hline! pour tracer
  une ligne. Par exemple:

\begin{verbatim}
   \revisions{
      indice & 1/10/2009 & statut & Création du document\\
      \hline
   } 
\end{verbatim}
\end{itemize}

\subsection{Documents avec visa}
\label{sec:documents-avec-visa}

Il arrive que certaines notes nécessitent un visa de la part d'un
approbateur. Dans ce cas, il suffit d'ajouter l'option \verb!visa!
lors du chargement de la classe. Dans ce cas, on dispose de nouvelles
commandes:

\begin{itemize}
\item la commande \verb!\authorsig! permet de donner une éventuelle
  signature du rédacteur sous forme d'image en utilisant
  \verb!\includegraphics! par exemple
\item la commande \verb!\visaname! permet de donner le nom de
  l'approbateur
\item la commande \verb!\visafunction! permet de donner la fonction de
  l'approbateur
\item la commande \verb!\paperorig! permet de signaler que l'original
  existe sous forme papier
\item la commande \verb!\filesrc! permet de signaler qu'il existe une
  source informatique
\item la commande \verb!\filename! permet de donner le nom de fichier
  de la source informatique
\end{itemize}

\subsection{Fichier de déclarations}
\label{sec:fich-de-decl}

Il se peut que l'on utilise souvent les mêmes destinataires etc. On
peut avec la classe \texttt{supaero-note} utiliser un fichier
\texttt{mon-fichier.sup} qui contient des redéfinitions (par exemple
\verb!\author{}! etc.). Pour cela, il suffit de passer
\verb!mon-fichier! en option de \verb!\documentclass!:

\begin{verbatim}
  \documentclass[mon-fichier]{supaero-note}
\end{verbatim}

\section{Code source de la présente note}
\label{sec:code-source}

Le code source de la présente note est présenté sur le listing
suivant.

\verbatiminput{exempleNote.tex}

\section{Code source de la classe}
\label{sec:code-source-supaero-note}

\verbatiminput{supaero-note.cls}

\printbibliography

\end{document}
