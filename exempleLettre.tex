\documentclass[pdftex,12pt,a4paper,origdate]{lettre}

\usepackage[utf8]{inputenc}
\usepackage[french]{babel}
\usepackage[T1]{fontenc}
\usepackage{ae,aecompl,aeguill}
\usepackage{tikz}
\usepackage{amsmath}
\usepackage[pdftex]{hyperref}


\DeclareOldFontCommand{\rm}{\sffamily}{\mathsf}
\renewcommand\familydefault{cmss}

%======================================================================
% Début du document
%======================================================================
\begin{document}

%======================================================================
% Le nom pour la signature et le destinataire
%======================================================================
\name{Christophe Garion\\Resp. «~Mise à niveau littéraires~»}

\institut{supaero}

\begin{letter}{Ausias Gamisans\\ Professeur Associé d'Espagnolo
    Facilo\\ISAE/LACS}

%======================================================================
% Adresse différente, pas de téléphone etc.
%======================================================================
%\address{}
%\telephone{}
%\nofax
%\email{}
%lieu{}
%\Nref{}
%\Vref{}
%\conc{}

\opening{Cher Ausias,}

Suite à ta question d'hier, je me permets de t'écrire pour te rappeler
la formulation du théorème de Pythagore. Supposons\footnote{Les
  scientifiques utilisent beaucoup ce mot. C'est pour poser ce que
  l'on appelle une \emph{hypothèse}.} que l'on ait un triangle
rectangle de côtés de longueurs $a$, $b$ et $c$ (les côtés
adjacents\footnote{Adjacent signifie «~à côté de ~». Si si,
  souviens-toi, tu as entendu ce mot au collège.} à l'angle droit
étant ceux de longueur $a$ et $b$) comme représenté ci-dessous:

  \begin{tikzpicture}
    \filldraw[draw=blue!50!black, fill=blue!20] (0,0) -- node[pos=0.5,
    color=black, below, yshift=-2pt] {$a$} (4,0) -- node[pos=0.5, color=black,
    right, xshift=4pt, anchor=west] {$c$} (0,3) -- node[color=black, pos=0.5, left, xshift=-2pt] {$b$} (0,0);
  \end{tikzpicture}

  Dans ce cas, l'équation suivante, appelée théorème de Pythagore,
  permet de relier les trois grandeurs $a$, $b$ et $c$:

    \begin{equation}
      \label{eq:1}
      a^2 + b^2 = c^2
    \end{equation}

Dans notre cas, si $a = 4$ et $b = 3$ et que l'on cherche la valeur de
$c$, on peut donc écrire\footnote{Tu peux mesurer la figure précédente
  avec ta règle graduée et trouver la solution, mais ce n'est pas
  l'objectif de l'exercice.}:

\begin{equation*}
  \begin{split}
    c^2 & = 4^2 + 3^2\\
     & = 16 + 9\\
     & = 25\\
    c & = \sqrt{25} \text{ car $c$ est une valeur
      positive\footnotemark}\\
      & = 5
  \end{split}
\end{equation*}

\footnotetext{Souviens-toi, il s'agit d'une longueur!}

Bonus vocabulaire: le côté de longueur $c$ s'appelle
\emph{l'hypoténuse}. Ce n'est pas vrai pour tous les côtés de longueur
$c$, il s'agit en fait du côté qui ne «~touche~» pas l'angle
droit. Dans notre exemple il est de longueur $c$, mais c'est un cas
particulier.

La semaine prochaine, nous parlerons des fractions. C'est un peu
difficile, mais je sens que tu as le potentiel pour bien comprendre
tout cela.

\closing{Amitiés scientifiques,}

%\encl{}

\end{letter}
\end{document}
