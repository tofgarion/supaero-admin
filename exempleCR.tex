\documentclass[fr,cr]{supaero-mins}

\usepackage[utf8]{inputenc}
\usepackage{url}
\usepackage{verbatim}

\begin{document}

\title{Réunion entre C. Garion et lui-même sur \LaTeX}
\author{C. Garion}
\date{\today}

\present{C. Garion}
\absent{D. Matignon, E. Zenou}
\alsopresent{S. Garion (pour embêter son père quand il travaille)}

\maketitle

\begin{objectifs}
\item proposer une classe \LaTeX permettant d'écrire facilement des
  comptes-rendus de réunions
\item écrire un petit exemple montrant les possibilités offertes par
  la classe
\item éradiquer Word comme outil de rédaction de compte-rendus
\item perdre mon temps alors que j'ai d'autres chats à fouetter (aïe)
\end{objectifs}

\begin{crlist}{Chargement de la classe et dimensions}
\item la classe s'appelle \texttt{supaero-mins}\footnote{mins pour
    minutes, en anglais dans le texte. C'est aussi parce que la classe
  dont je me suis inspiré porte ce nom.}
\item les dimensions des pages sont réglées grâce au paquetage
  \texttt{geometry}. On pourra se référer à sa documentation pour
  l'utiliser.
\item les options suivantes sont disponibles pour la classe:
  \begin{itemize}
  \item \verb!en! pour choisir un document en anglais (choisi par
    défaut)
  \item \verb!fr! pour choisir un document en français
  \item \verb!en-fr! pour choisir un document en anglais et en
    français, avec la plupart du document en anglais (utiliser la
    commande \verb!\setlanguage! pour changer la langue)
  \item \verb!fr-en! pour choisir un document en français et en
    anglais, avec la plupart du document en anglais
  \item \verb!biblatex! pour choisir \verb!biblatex! comme outil de
    gestion de la bibliographie
  \end{itemize}
\end{crlist}

\begin{crlist}{Entêtes et pieds de page}
\item les entêtes et pieds de page sont définis grâce au paquetage
  \texttt{fancyhdr}. On dispose donc des commandes habituelles de ce
  paquetage pour les redéfinir.
\end{crlist}

\begin{crlist}{Page de titre et commandes utiles}
\item la page de titre est définie grâce aux éléments suivants:

  \begin{itemize}
  \item la commande \verb!\title{}! permet de donner le titre. On peut
    lui donner en option un titre court qui servira sur les entêtes
    des pages suivantes (par exemple
    \verb!\title[titrecourt]{titresuperlong}!)
  \item la commande \verb!\author{}! permet de donner le nom du
    rédacteur du document
  \item la commande \verb!\date{}! permet de donner la date de la
    réunion
  \item la commande \verb!\members{}! permet de donner les personnes
    qui étaient conviées à la réunion
  \item la commande \verb!\present{}! permet de donner la liste des
    personnes conviées présentes
  \item la commande \verb!\absent{}! permet de donner la liste des
    personnes conviées absentes
  \item la commande \verb!\alsopresent{}! permet de donner la liste
    des personnes présentes et qui n'étaient pas conviées au départ
  \end{itemize}

\item la commande \verb!mintitle! permet de donner un titre à une
  «~section~» du compte-rendu.
\item on dispose également d'un environnement \verb!crenum! qui
  s'utilise comme \verb!enumerate! en donnant en plus un titre pour la
  «~section~».
\item on dispose également d'un environnement \verb!crlist! qui
  s'utilise comme \verb!itemize! en donnant en plus un titre pour la
  «~section~».
\item enfin, \verb!crlist! a été utilisé pour créer deux
  environnements: \verb!objectifs! et \verb!actions!.
\end{crlist}

\begin{crlist}{Fichier de déclaration}
\item lorsque l'on utilise souvent la même définition pour une
  commande (\verb!author! typiquement), on peut définir tout cela dans
  un fichier séparé possédant l'extension \verb!.sup!.
\item il suffit alors de passer en option de \verb!documentclass! le
  nom de ce fichier sans extension (cf. présent document par exemple).
\end{crlist}

\mintitle{Code source de la présente note}

\verbatiminput{exempleCR.tex}

\mintitle{Code source de la classe}

\verbatiminput{supaero-mins.cls}

\begin{actions}
\item C. Garion diffuse la classe et les exemples
\item les personnes intéressées font des retours éventuels sur la
  classe
\end{actions}
\end{document}
